\section{Introduction}
\label{sec:intro}
The Schrödinger equation allows us to express the energy of a wave function. However, since it is not exactly solvable for systems that 
have more than one electron, e.i. all most all things of importance, we are forced to make approximations. In these approximations we will
deviate from reality by applying certain constraints that will allow us to solve the system. However, doing so will decrease the amount
of freedom a system has to evolve. One might ask the question wheter we are not artifficially creating something that behaves like it is 
reality, but acctually is nothing more than a construct. However if we do not impose constraints, we lose a lot of control, meaning that 
interpretation and data collection might jeopardized. 

In this particular case we will take a look at the symmetry in the Hamiltonian. We know that in molecules there can be a lot of symmetry.
This symmetry also exists in that molecules Hamiltonian, and thus also in every eigenfunction. Now we can easily state that every 
approximation we make of these eigenfunctions must contain that same symmetry. If we aspire to create somthing that is close to reality,
but end up with something that does not have any real symmetry, we can do better. The important relation here is that the Hamiltonian 
commutes with the $S^2$ operator, meaning that these operators share a complete set of eigenfunctions. Thus an eigenfunction of the Hamiltonian must also
be an eigenfunction of $S^2$. This operator is responsible for the symmetry in a molecule.

In this paper we will discuss several modes in Hartree-Fock theory, focussed specifically on this symmetry. We will look at Restricted Closed
Shell HF, Unrestricted HF, Constrianed Unrestricted HF and in a minor capacity at Restricted Open Shell HF. We will compare the energies
and wave functions and we will discuss using a symmetry argument why some energy might be higer than another. We will use 
Single excitation Configuration Interaction (CIS) to help in this analysis, as it will allow us to see multiplet excitation energies, 
which will indicate certain symmetries. If at all possible we will also attempt to visualise the orbitals obtained. 

\section{Hartree Fock Modes}
In this section we will discuss the various Hartree Fock modes we have analysed over the course of this Bachelor project. For UHF and RHF we will not go into detail.
CUHF will be discussed a little more rigourous, but not to the same level of detail as was done last year. ROHF will be discussed, but very superficial, since it is
only needed as a comparative method. For every mode we will attempt to focus on the part that talks about symmetry. The final goal of this section then would be to
explain for each mode where the symmetry comes from or why it is no longer there.

\section{Single Excitation Configuration Interaction}
The theory will be explained in detail, so that we can establish a good understanding of what it is and why we might want to do it. The focus will be on symmetry.
We will also explain Brillouin's theorem. The question I would like to answer here is "Why are there triplets showing up in the excited states?"

\section{results}
I am not yet sure about the precise oder here, but this section is divided in two parts.

\subsection{$H_2$ stretch}
Where we will display the plots of $H_2$ for all the modes, plotting spin contamination as well.

\subsection{Excitations}
Where we will discuss the datafiles as generated by the CIS molecule object. We would make plots of the varoius excitation energies, given that they are correct and
I can figure out how to plot degenerate levels next to each other.

\subsection{orbital diagrams}
Given that I want to focus on symmetry, it might be nice to draw up orbitals. However I currently have know idea how.

\section{conclusion}
CUHF retains the symmetry, as can be seen in the datafiles. 
CUHF give differnet excitation energies than ROHF (has yet to be confirmed)

\section{loose ends}
I am not sure where to fit these PassOptionsToPackage
\begin{enumerate}
    \item Determine point group for target molecule (allyl) and try to fit it into the results.
\end{enumerate}
