%%%%%%%%%%%%%%%%%%%%%%%%%%%%%%%%%%%
%This is the LaTeX ARTICLE template for RSC journals
%Copyright The Royal Society of Chemistry 2016
%%%%%%%%%%%%%%%%%%%%%%%%%%%%%%%%%%%

\documentclass[twoside,twocolumn,9pt]{article}
\usepackage{tgheros}
\usepackage[utf8]{inputenc}
\renewcommand{\familydefault}{\sfdefault}

\usepackage{extsizes}
\usepackage[super,sort&compress,comma]{natbib} 
\usepackage[version=3]{mhchem}
\usepackage[left=1.5cm, right=1.5cm, top=1.785cm, bottom=2.0cm]{geometry}
\usepackage{balance}
\usepackage{mathptmx}
\usepackage{sectsty}
\usepackage{graphicx} 
\usepackage{lastpage}
\usepackage[format=plain,justification=justified,singlelinecheck=false,font={stretch=1.125,small,sf},labelfont=bf,labelsep=space]{caption}
\usepackage{float}
\usepackage{fancyhdr}
\usepackage{fnpos}
\usepackage[english, german]{babel}
\addto{\captionsenglish}{%
  \renewcommand{\refname}{Notes and references}
}
\usepackage{array}
\usepackage{droidsans}
\usepackage{charter}
\usepackage[T1]{fontenc}
\usepackage[usenames,dvipsnames]{xcolor}
\usepackage{setspace}
\usepackage[compact]{titlesec}
%%%Please don't disable any packages in the preamble, as this may cause the template to display incorrectly.%%%


\usepackage{xcolor}
\definecolor{ugent_blue}{RGB}{30, 100, 200}
\definecolor{ugent_yellow}{cmyk}{.0, .10, 1, 0}

\usepackage{titlesec}
\titleformat{\section}
{\color{ugent_blue}\normalfont\Large\bfseries}
{\color{ugent_blue}\thesection}{1em}{}

\usepackage[colorlinks=true,linkcolor=black,citecolor=ugent_blue]{hyperref}

%\AtEveryCite{\color{ugent_blue}}




\usepackage{epstopdf}%This line makes .eps figures into .pdf - please comment out if not required.

\definecolor{cream}{RGB}{222,217,201}

\begin{document}

\pagestyle{fancy}
\thispagestyle{plain}
\fancypagestyle{plain}{
%%%HEADER%%%
\renewcommand{\headrulewidth}{0pt}
}
%%%END OF HEADER%%%

%%%PAGE SETUP - Please do not change any commands within this section%%%
\makeFNbottom
\makeatletter
\renewcommand\LARGE{\@setfontsize\LARGE{15pt}{17}}
\renewcommand\Large{\@setfontsize\Large{12pt}{14}}
\renewcommand\large{\@setfontsize\large{10pt}{12}}
\renewcommand\footnotesize{\@setfontsize\footnotesize{7pt}{10}}
\makeatother

\renewcommand{\thefootnote}{\fnsymbol{footnote}}
\renewcommand\footnoterule{\vspace*{1pt}% 
\color{cream}\hrule width 3.5in height 0.4pt \color{black}\vspace*{5pt}} 
\setcounter{secnumdepth}{5}

\makeatletter 
\renewcommand\@biblabel[1]{#1}            
\renewcommand\@makefntext[1]% 
{\noindent\makebox[0pt][r]{\@thefnmark\,}#1}
\makeatother 
\renewcommand{\figurename}{\small{Fig.}~}
\sectionfont{\sffamily\Large}
\subsectionfont{\normalsize}
\subsubsectionfont{\bf}
\setstretch{1.125} %In particular, please do not alter this line.
\setlength{\skip\footins}{0.8cm}
\setlength{\footnotesep}{0.25cm}
\setlength{\jot}{10pt}
\titlespacing*{\section}{0pt}{4pt}{4pt}
\titlespacing*{\subsection}{0pt}{15pt}{1pt}
%%%END OF PAGE SETUP%%%

%%%FOOTER%%%
\fancyfoot{}
\fancyfoot[LO,RE]{\vspace{-7.1pt}\includegraphics[height=9pt]{head_foot/LF}}
\fancyfoot[CO]{\vspace{-7.1pt}\hspace{11.9cm}\includegraphics{head_foot/RF}}
\fancyfoot[CE]{\vspace{-7.2pt}\hspace{-13.2cm}\includegraphics{head_foot/RF}}
\fancyfoot[RO]{\footnotesize{\sffamily{1--\pageref{LastPage} {\color{ugent_yellow} ~\textbar } \hspace{2pt}\thepage}}}
\fancyfoot[LE]{\footnotesize{\sffamily{\thepage~{\color{ugent_yellow} ~\textbar }\hspace{4.65cm} 1--\pageref{LastPage}}}}
\fancyhead{}
\renewcommand{\headrulewidth}{0pt} 
\renewcommand{\footrulewidth}{0pt}
\setlength{\arrayrulewidth}{1pt}
\setlength{\columnsep}{6.5mm}
\setlength\bibsep{1pt}
%%%END OF FOOTER%%%

%%%FIGURE SETUP - please do not change any commands within this section%%%
\makeatletter 
\newlength{\figrulesep} 
\setlength{\figrulesep}{0.5\textfloatsep} 

\newcommand{\topfigrule}{\vspace*{-1pt}% 
\noindent{\color{cream}\rule[-\figrulesep]{\columnwidth}{1.5pt}} }

\newcommand{\botfigrule}{\vspace*{-2pt}% 
\noindent{\color{cream}\rule[\figrulesep]{\columnwidth}{1.5pt}} }

\newcommand{\dblfigrule}{\vspace*{-1pt}% 
\noindent{\color{cream}\rule[-\figrulesep]{\textwidth}{1.5pt}} }

\makeatother
%%%END OF FIGURE SETUP%%%

%%%TITLE, AUTHORS AND ABSTRACT%%%
\twocolumn[
  \begin{@twocolumnfalse}
{\includegraphics[height=30pt]{head_foot/journal_name}\hfill\raisebox{0pt}[0pt][0pt]{\includegraphics[height=55pt]{head_foot/RSC_LOGO_CMYK}}\\[1ex]
\includegraphics[width=18.5cm]{head_foot/header_bar}}\par
\vspace{1em}
\sffamily
\begin{tabular}{m{4.5cm} p{13.5cm} }

& \noindent\LARGE{\textbf{This is the title$^\dag$}} \\%Article title goes here instead of the text "This is the title"
\vspace{0.3cm} & \vspace{0.3cm} \\

& \noindent\large{Full Name,$^{\ast}$\textit{$^{a}$} Full Name,\textit{$^{b\ddag}$} and Full Name\textit{$^{a}$}} \\%Author names go here instead of "Full name", etc.

& \\

& \noindent\normalsize{Do \emph{not} write an abstract. That will be done when the outline has matured into a completed paper.} \\%The abstrast goes here instead of the text "The abstract should be..."

\end{tabular}

\end{@twocolumnfalse} \vspace{1.6cm}

  ]
%%%END OF TITLE, AUTHORS AND ABSTRACT%%%

%%%FONT SETUP - please do not change any commands within this section
\renewcommand*\rmdefault{bch}\normalfont\upshape
\rmfamily
\section*{}
\vspace{-1cm}


% %%%FOOTNOTES%%%

\footnotetext{\textit{$^{a}$~Ghent Quantum Chemistry Group, Krijgslaan 281 (S3), B-9000 Gent, België}}
\footnotetext{\textit{$^{b}$~Corresponding author:} \texttt{firstname.lastname@ugent.be}}

% %Please use \dag to cite the ESI in the main text of the article.
% %If you article does not have ESI please remove the the \dag symbol from the title and the footnotetext below.
% \footnotetext{\dag~Electronic Supplementary Information (ESI) available: [details of any supplementary information available should be included here]. See DOI: 00.0000/00000000.}
% %additional addresses can be cited as above using the lower-case letters, c, d, e... If all authors are from the same address, no letter is required

% \footnotetext{\ddag~Additional footnotes to the title and authors can be included \textit{e.g.}\ `Present address:' or `These authors contributed equally to this work' as above using the symbols: \ddag, \textsection, and \P. Please place the appropriate symbol next to the author's name and include a \texttt{\textbackslash footnotetext} entry in the the correct place in the list.}


%%%END OF FOOTNOTES%%%

%%%MAIN TEXT%%%%


\section{Introduction}

Central ideas: Why did I do the work? What were the central motivations and hypotheses?

\paragraph*{}
The Shrödinger equation allows us to express the energy of a wave function. However, since it is not exactly solvable for systems that 
have more than one electron, e.i. all most all things of importance, we are forced to make approximations. In these approximations we will
deviate from reality by applying certain constraints that will allow us to solve the system. However, doing so will decrease the amount
of freedom a system has to evolve. One might ask the question wheter we are not artifficially creating something that behaves like it is 
reality, but acctually is nothing more than a construct. However if we do not impose constraints, we lose a lot of control, meaning that 
interpretation of data might be jeopardized. 
\paragraph*{}
In this particular paper we will take a look at the symmetry in the Hamiltonian. We know that in molecules there can be a lot of symmetry.
This symmetry also exists in that molecules Hamiltonian, and thus also in every eigenfunction. Now we can easily state that every 
approximation we make of these eigenfunctions must contain that same symmetry. If we aspire to create somthing that is close to reality,
but end up with something that does not have any real symmetry, we can do better. The important relation here is that the Hamiltonian 
commutes with the $S^2$ operator, meaning that these operators share a complete set of eigenfunctions. Thus an eigenfunction of the Hamiltonian must also
be an eigenfunction of $S^2$. This operator is responsible for the symmetry in a molecule.


\begin{itemize}
    \item The objectives of the work. 
    \paragraph*{}
    We want to expand the framework of Constrained Unrestricted Hartree Fock theory by applying Single Excitation Configuration 
    Interaction to the CUHF wave function. Is CUHF a good reference function for CIS? 
    \item The justification for these objectives: Why is the work important? Cite the most important works \cite{whitesides2004a}.
    \begin{enumerate}
      \item If CUHF is proven to be a good reference for CIS, we might be able to take it further (in more involved CI calculations).
      \item Plakhutin and Davidson explained what ionisations are not allowed in CUHF. They do not go into detail on excitations though. We will explore this area in more detail.
      \item We can not lose track of the symmetry. In a presentation by Scureria, he states that UHF has the right energy, yet the wrong wave function.
    \end{enumerate}
    \item Background: Who else has done what? How? What have we done previously?
    \paragraph*{}
    \begin{enumerate}
      \item Tsuchimochi and Scuseria \textit{Communication: ROHF theory made simple}: this describes CUHF theory
      \item Plakhutin and Davidson \textit{Canonical form of the Hartree-Fock orbitals in open-shell systems}: describes what is wrong with CUHF, comes with energy diagrams
      \item Tsuchimochi and Scuseria \textit{Constrained active space unrestricted mean-field methods for controlling spin-contamination}: a more detailed description of CUHF, with very interesting figures.
      \paragraph*{}
      Tsuchimochi and Scuseria are the creators of CUHF method. In their papers they explain the theory and apply it to some model systems. Plakhutin and Davidson describe different HF methods including CUHF and compare them to ROHF. They focus on Koopman's theorem and why it applies/does not apply to a certain method.
    \end{enumerate}
    \item Guidance to the reader: What should the reader watch for in the paper? What are the interesting high points? What strategy did we use?
    \paragraph*{}
    In this paper we will discuss several modes in Hartree-Fock theory, focussed specifically on this symmetry. We will look at Restricted Closed
Shell HF, Unrestricted HF, Constrained Unrestricted HF and in a minor capacity at Restricted Open Shell HF. We will compare the energies
and wave functions and we will discuss using a symmetry argument why some energy might be higer than another. We will use 
Single excitation Configuration Interaction (CIS) to help in this analysis, as it will allow us to see multiplet excitation energies, 
which will indicate certain symmetries. If at all possible we will also attempt to visualise the orbitals obtained. 
    \item Summary/conclusion: What should the reader expect as conclusion? In advanced versions of the outline, you should also include all the sections that will go in the Experimental section (at the level of paragraph subheadings) and indicate what information will go in the Microfilm section.
    \paragraph*{}
    The CUHF wave function has the same symmetry we expect as the true wave function for restricted systems. It should give the same results as ROHF. However, as Plakhutin and Davidson point out, we can expect CUHF to fail for certain excitations.
\end{itemize}

\section{Theory}

Optional section where you explain any theoretical constructs. In preliminary outline phases, this can be reduced to a set of essential equations.
\paragraph*{Concepts}
See notes on theory.
\begin{enumerate}
  \item RHF: The RHF wave function is not spin contaminated. This means that it is an eigenfunction of the $S^2$ operator, and it will follow the symmetry.
  \item UHF: The UHF wave function is spin contaminated. This means that it will not follow the correct symmetry.
  \item CUHF: The wave function is not spin contaminated.
  \item ROHF: the wave function is not spin contaminated.
  \item CIS
\end{enumerate}
\paragraph*{Equations to use}
\begin{enumerate}
  \item spin contamination general (1.6)
  \item spin contamination in NO basis (2.9) => DD (detailed dervation)
  \item expectation value over H for CIS (3.9) => DD
\end{enumerate}
\paragraph*{References}
\begin{enumerate}
  \item Andrews et al. \textit{Spin contamination in single-determinant wavefunctions}
  \item Tsuchimochi and Scuseria \textit{Communication: ROHF theory made simple}: this describes CUHF theory
  \item Sherrill \textit{Derivation of the Configuration Interaction Singles (CIS) Method for Various Single Determinant References and Extensions to Include Selected Double Substitutions (XCIS)}
\end{enumerate}

\section{Methodology}

How were methods/descriptors implemented and characterized? Which systems are of interest for this hypothesis?
\paragraph*{}
The $H_3$ radical will be used as a model system. This is a small molecule that allows for easy interpretation and oversight. We will also use a limited basis, sto-3g, for these same reasons.
\section{Results and discussion}

Central ideas: What were the results? Organize the outline and the paper around easily assimilated data - tables, equations, figures, schemes - rather than around text. Organize in order of importance, not in chronological order.

\begin{itemize}
    \item The $H_2$ stretch in different HF modes.
    \item The excited states in different HF modes
    \item Spin contamination in the excited states
\end{itemize}

\subsection{$H_2$ Stretch}
We will demonstrate the effect of spin contamination using the stretch of the dihydrogen molecule.
\begin{center}
  \begin{figure}[h]
      \includegraphics[width=\linewidth]{./../notes/figures/rhf.png}
      \caption{The RHF energy against the distance in an $H_2$ stretch. The zero level was chosen as the energy of two separate hydrogen atoms.}
      \label{fig:rhfstretch}
  \end{figure}
\end{center}
In Figure \ref{fig:rhfstretch} we can see that the RHF solution is not physically valid. Indeed, we expect the energy to evolve towards the energy of two seperate hydrogen atoms, which clearly does not happen. However we see that the spin contamination is zero, so the wave function has the correct symmetry. The reason for this is simple. We artificially trap the electrons in the same orbital.
\begin{center}
\begin{figure}[h]
  \includegraphics[width=\linewidth]{./../notes/figures/uhf.png}
  \caption{The UHF energy against the distance in an $H_2$ stretch}
  \label{fig:uhfstretch}
\end{figure}
\end{center}
Figure \ref{fig:uhfstretch} shows a physically more correct result. We see that the system does evolve towards two hydrogen atoms. Howver, the spin contamination is rising, and after a certain point reaches the value one. This means that the wave function does no longer have the right symmetry, ergo it is no longer the right wave function.
\begin{center}
\begin{figure}[h]
  \includegraphics[width=\linewidth]{./../notes/figures/cuhf_mix.png}
  \caption{The CUHF energy against the distance in an $H_2$ stretch}
  \label{fig:cuhfstretch}
\end{figure}
\end{center}
The CUHF picture in Figure \ref{fig:cuhfstretch} is difficult to interpret. We see that the spin contamination remains zero. The energy seems to follow the RHF pattern (see also Figure \ref{fig:combo}) untill a certain point. Then it starts to exponantially decline. Physically this seems like a logical behaviour, as it is now again eveolving to two separate hydrogen atoms. 
\begin{center}
\begin{figure}[h]
  \includegraphics[width=\linewidth]{./../notes/figures/combo.png}
  \caption{The stretch energies plotted together}
  \label{fig:combo}
\end{figure}
\end{center}
For comparison we plotted all the stretches in one plot, Figure \ref{fig:combo}. Here we can see that they all describe short bond lengths correctly, but as the internuclear distance increases, there are deviations visible.
\subsection{CIS results}
We need to plot the excitation energies of $H_3$, together with the ground state energy. This will be done for UHF, CUHF and ROHF, as soon as we can verify our data.
\begin{itemize}
    \item Method implementation
    \paragraph*{}
    For the code we used python with the help of the psi4 package.
    \item Method characterization
    \item System characterization
    \item Results
\end{itemize}
\paragraph*{Figures}
\begin{itemize}
    \item Excitation energies in all modes 
    \item orbitals?
\end{itemize}
\paragraph*{Schemes}
\begin{itemize}
  \item electron configurations for different modes (learn to draw them!!)
\end{itemize}



\section{Conclusions}

Central ideas: What does it all mean? What hypotheses were proved or disproved? What did I learn? Why does it make a difference?


%%%END OF MAIN TEXT%%%

%The \balance command can be used to balance the columns on the final page if desired. It should be placed anywhere within the first column of the last page.

\balance

%If notes are included in your references you can change the title from 'References' to 'Notes and references' using the following command:
%\renewcommand\refname{Notes and references}

%%%REFERENCES%%%
\bibliography{outline} %You need to replace "rsc" on this line with the name of your .bib file
\bibliographystyle{aip} %the AIP's .bst file

\end{document}
