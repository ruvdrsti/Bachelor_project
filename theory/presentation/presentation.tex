% Aspectratio 16:9 should be used
% The theme is not suited for 4:3 aspectratio
\documentclass[aspectratio=169]{beamer}

\usepackage{mhchem}
% Metadata of the presentation
\title{\ce{H3} as a model for open shell systems}
\subtitle{CIS with CUHF and UHF references}
\date[ISBT 2018]{Bachelor project}
\author[DB]{Ruben Van der Stichelen}

% Macro aimed at loading themes in different directories
\makeatletter
  \def\beamer@calltheme#1#2#3{%
    \def\beamer@themelist{#2}
    \@for\beamer@themename:=\beamer@themelist\do
    {\usepackage[{#1}]{\beamer@themelocation/#3\beamer@themename}}}

  \def\usefolder#1{
    \def\beamer@themelocation{#1}
  }
  \def\beamer@themelocation{}

% Load the UGent theme
\usefolder{theme}
\usetheme[language=en,faculty=we,usecolors]{ugent}
\useinnertheme{ugent}
\useoutertheme{ugent}
\usecolortheme{ugent}
\usefonttheme{ugent}

% Path to images
\graphicspath{{theme/}}

% Have this if you'd like section slides 
\AtBeginSection[]{
    \sectionframe
}

\begin{document}

% Have this if you'd like the presentation to start 
% with a large UGent logo
\logoframe

% I guess you always want a titleframe
\titleframe

% Have this if you'd like a frame containing the 
% table of content
\begin{frame}{Overview}
    \tableofcontents[hideallsubsections]
\end{frame}

% Start of the first section
\section{Introduction}

\begin{frame}
    \frametitle{The Hartree-Fock approximation}
    \begin{equation}
        \hat{H}\Psi = E\Psi
    \end{equation}
    \begin{equation}
        \hat{f} = \hat{h}(1) + \sum_j^N\hat{J}_j - \hat{K}_j
    \end{equation}
    \begin{equation}
        \langle \psi + \delta\psi | \hat{H} | \psi + \delta\psi \rangle = E + \delta E^* + \delta E + \delta E^2
    \end{equation}
\end{frame}

\begin{frame}
    \frametitle{Löwdin's symmetry dillema}
    \begin{itemize}
        \item Restricted Hartree-Fock
        \begin{itemize}
            \item Higher energy
            \item Symmetry intact
        \end{itemize}
        \item Unrestricted Hartree-Fock
        \begin{itemize}
            \item Lower energy
            \item Spin quantum numbers lost
        \end{itemize}
    \end{itemize}

\end{frame}


\section{Theory}

\begin{frame}
    \frametitle{Spin contamination}
    \begin{equation}
        \langle \hat{S}^2 \rangle = S_z^2 + S_z + q - \sum_{ij}^{pq}S_{ij}^2
    \end{equation}

    \begin{equation}
        \delta s = \langle \hat{S}^2 \rangle - S_z(S_z + 1)
    \end{equation}

    \begin{equation}
        \delta s = 4\sum^{N_cp}_i \sqrt{(n_i^2 - n_i)}
    \end{equation}

\end{frame}

\begin{frame}
    \frametitle{Constrained unrestricted Hartree-Fock theory}
    \begin{equation}
        \tilde{F}^{\sigma} = F_{cs} \pm \Delta^{CUHF}
    \end{equation}
    where
    \begin{equation}
        \Delta^{CUHF} = \begin{cases}
            0, & \mbox{in the vc blocks} \\
            \Delta^{UHF}, & \mbox{anywhere else}
        \end{cases}
    \end{equation}
    $\Longrightarrow$ transformations needed
\end{frame}

\begin{frame}
    \frametitle{Configuration interaction}
    \begin{equation}\label{eq:Ecorr}
        E_{corr} = \mathcal{E}_0 - E_0
      \end{equation}     
    \begin{equation}\label{eq:lincomb}
        |\Psi_0\rangle = c_0|\Phi_0\rangle + \sum_{ar}|\Phi_a^r\rangle + \sum_{a<b,r<s}c_{ab}^{rs}|\Phi^{rs}_{ab} \rangle + \cdots
      \end{equation}
      \begin{equation}\label{eq:overlapsingle}
        \langle\Phi_0 |\hat{H}|\Phi_r^a\rangle = \langle a|h|r \rangle + \sum_b \langle ab||rb \rangle = \langle \chi_a |\hat{f}| \chi_r \rangle = F_{ra}
      \end{equation}
      \begin{equation}\label{eq:matrixelement}
        \langle \Phi_r^a|\hat{H}|\Phi_s^b \rangle = E_0\delta_{rs}\delta_{ab} + F_{ab}\delta_{rs} - F_{rs}\delta_{ab} + \langle as || jb \rangle
      \end{equation}
\end{frame}


\section{\ce{H2} and \ce{H3} stretches}


\begin{frame}
    \frametitle{\ce{H2}}
\end{frame}

\begin{frame}
    \frametitle{\ce{H3}}
\end{frame}


\section{CIS results}

\begin{frame}
    \frametitle{Triplets}

\end{frame}

\begin{frame}
    \frametitle{orbitals}
\end{frame}

\begin{frame}
    \frametitle{$\hat{S}^2$ expectation values}
\end{frame}
\section{Conclustions and prospects}
\begin{frame}
    \begin{itemize}
        \item Triplet states 
        \item Deal with spin contamination in excited states
        \item Full CI
        \item Triplet systems
    \end{itemize}
    
\end{frame}

% End presentation with titleframe
\titleframe

\end{document}
