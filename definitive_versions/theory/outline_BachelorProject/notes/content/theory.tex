\section{Theory}

\subsection{Configuration Interaction}

The theory in this section is based on:
\begin{itemize}
    \item Szabo and Ostlund, \textit{Modern Quantum Chemistry, Introduction to Advanced Electronic Stucture Theory}
    \item Sherril, \textit{Derivation of the Configuration Interaction Singles (CIS) Method for Various Single Determinant References and Extensions to Include Selected Double Substitutions (XCIS)}
\end{itemize}

\paragraph{Post Hartree Fock methods}
The Hartree Fock approximation gives us an initial guess for the energy and the wavefunction. However, it can be improved upon. These methods are called post Hartree-Fock methods. The difference in energy between the two methods is called the correlation energy \eqref{eq:Ecorr}.
\begin{equation}\label{eq:Ecorr}
    E_{corr} = \mathcal{E}_0 - E_0
\end{equation}
Here $E_0$ is the Hartee-Fock energy and $\mathcal{E}_0$ it the post Hartree-Fock energy.

\paragraph{Configuration Interaction}
If $\Psi_0$ is a good approximation of the wave function $\Phi_0$, we know that we can make a better approximation using a linear combination of excited states \eqref{eq:lincomb}.
\begin{equation}\label{eq:lincomb}
    |\Phi_0\rangle = c_0|\Psi_0\rangle + \sum_{ar}|\Psi_a^r\rangle + ...
\end{equation}
In CIS we are only interested in the first order excitations. We can prove Now we want to calculate $\langle\Phi_0|\hat{H}|\Phi_0\rangle$ for which we will have to know some matrix elements, \eqref{eq:exp1}, \eqref{eq:exp2} and \eqref{eq:exp3}.
\begin{equation}\label{eq:exp1}
    \langle\Psi_0|\hat{H}|\Psi_0\rangle = E_{SCF}
\end{equation}
Trough Brillouin's theorem we know \eqref{eq:exp2}.
\begin{equation}\label{eq:exp2}
    \langle\Psi_0|\hat{H}|\Psi_a^r\rangle = 0
\end{equation}
And Sherril tells us \eqref{eq:exp3}.
\begin{equation}\label{eq:exp3}
    \langle\Psi_a^r|\hat{H}|\Psi_b^s\rangle = E_{SCF}\delta_{ij}\delta_{rs} + F_{rs}\delta_{ij} - F_{ij}\delta_{rs} + \langle rj||is\rangle
\end{equation}
At self consistent field conditions the Fock operator equals $F_{ij} = \epsilon_i\delta_{ij}$, so Equation \eqref{eq:exp3} can be simplified. If we now diagonilize this matrix we end up with the excitation energies and contributions of the single excited states as eigenfunctions. However, since electrons in alpha orbitals can be excited to beta orbitals, which is especcially important in UHF, we have to do some extra steps, using the Kronecker product \eqref{eq:kron}.
\begin{equation}\label{eq:kron}
    R' = I_2 \otimes (I_2 \otimes R)^T
\end{equation}
The $R'$ matrix now has double the dimensions of the original R matrix. This is necesarry for the two electron integrals, since they only contain half the orbitals. These integrals are in the original basis, but we need to account for all possible excitations, meaning that we have to account for alpha and beta orbitals at the same time. Since the two electron integrals are given in AO basis, they still need to be transformed to MO basis, since we want the Fock operators to be diagonal in Equation \eqref{eq:exp3}. To do this transformation we will need the coefficient matrix from Equation \eqref{eq:coefs}.
\begin{equation}\label{eq:coefs}
    C' = \begin{bmatrix}
        C_a & 0 \\
        0 & C_b \\
    \end{bmatrix}
\end{equation} 
In this matrix $C_a$ corresponds to the coefficient matrix of the alpha orbitals and $C_b$ to the coeficient matrix of the beta orbitals. The diagonal blocks are all zero, since there is no contibution of the alpha orbitals to the beta orbitals and vice versa.
